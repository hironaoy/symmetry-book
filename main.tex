\documentclass{article}
\usepackage{appendix}
\usepackage{graphicx} % Required for inserting images
\usepackage{graphics}
\usepackage[margin = 25mm]{geometry}
\usepackage{mathrsfs, mathtools}
\usepackage{amsmath, amsfonts, amsthm, amssymb}
\usepackage{bm}
\usepackage{physics}
\usepackage{tensor}
\usepackage{comment}
\usepackage{etoolbox}
\usepackage{tikz-cd}
\usepackage{fancyhdr}
\usepackage{slashed}

%% \usepackage{lxfonts}

\theoremstyle{definition}
\newtheorem{theorem}{theorem}[section]
\newtheorem{definition}{Definition}[section]
\newtheorem{proposition}{Proposition}[section]

\numberwithin{equation}{section}
\renewcommand{\baselinestretch}{1.3}
\renewcommand{\mapsto}{\longmapsto}

\renewcommand{\appendixname}{Appendix}
\usepackage[dvipdfmx, colorlinks=true, linkcolor=black, urlcolor=black, citecolor=blue]
           {hyperref}
\usepackage{pxjahyper}
\usepackage{ulem}

\usepackage{titlesec}
\titleformat{\section}
            [hang]
            {\bfseries}
            {\thesection}
            {1em}
            {}
\titleformat{\subsection}
            [hang]
            {\bfseries}
            {\thesubsection}
            {1em}
            {}


\newcommand{\diag}{\mathrm{diag}\,}
\newcommand{\st}{\quad \mathrm{s.t.} \quad}
\newcommand{\ob}{\mathrm{Ob}\,}

\title{{\bf Notes: generalised symmetries \& related topics}}
\author{大阪大学大学院理学研究科物理学専攻 大和 寛尚}
\date{}

% \newcounter{defcount}
% \setcounter{defcount}{1}
% \renewenvironment{definition}{
%   \par
%   \vspace{2mm}
%   \noindent
%   \uline{Definition \thesection.\thedefcount}
% }{
%   \stepcounter{defcount}
%   \par
%   \vspace{2mm}
%   \aftergroup\noindent\ignorespacesafterend
% }

% \newcounter{propcount}
% \setcounter{propcount}{1}
% \renewenvironment{proposition}{
%   \par
%   \vspace{2mm}
%   \noindent
%   \uline{Proposition \thesection.\thepropcount}
% }{
%   \stepcounter{propcount}
%   \par
%   \vspace{2mm}
%   \aftergroup\noindent\ignorespacesafterend
% }

% \newenvironment{example}{
%   \par
%   \vspace{2mm}
%   \noindent
%   \uline{Example}
% }{
%   \par
%   \vspace{2mm}
%   \aftergroup\noindent\ignorespacesafterend
% }

\begin{document}
\maketitle
% \begin{abstract}
%   This notes are the record of my study on generalised symmetries. My plan is to cover every aspects of generalised symmetries, starting from its mathematical formulations (higher-form symmetries, categorical symmetries, categorical higher-form symmetries, higher gauge theories, etc.) to its applications to phenomenology (high density qunatum chromodynamics, beyond Ginzburg-Landau theories, etc.).
% \end{abstract}

\tableofcontents
\footnote{The sections with asterisk `` * '' are materials that is not directly related to generalised symmetries but supplemental.}

\section{Defining higher-form symmetries}
Most parts of this section are adopted from the reviews below.
\begin{enumerate}
\item [i.)] 日高 義将,
\uline{\href{https://ribf.riken.jp/~hidaka/yh/slide/hidaka_higher_form.pdf}{高次対称性入門}}
\item [ii.)] 日高 義将,
\uline{\href{https://ribf.riken.jp/~hidaka/yh/slide/hidaka_symmetry.pdf}{対称性の自発的破れ入門}}
\end{enumerate}

\subsection{Starting from the conventional symmetries}
Let us start by introducing another standpoint of seeing the conventional symmetries. We start with an example of a free fermionic system. The action of the system is given by
\begin{align}
  S[\psi] = \int_{\mathscr{M}} \dd^d x\, \bar{\psi}(x) \qty(\gamma^\mu\partial_\mu - m) \psi(x).
\end{align}
Here, $\mathscr{M}$ is a Minkovski manifold as spacetime and $\psi(x)$ is a spinor field, which is a section of a vector bundle $E \xrightarrow{\,\,\pi\,\,} \mathscr{M}$ with $\mathrm{Spin(1,3)}$ group as its structure group.

\section{Reviewing the renormalisation standpoint in quantum field theory *}
\section{Reviewing the spontaneous symmetry breaking of conventional symmetries *}
\section{Generalising the spontaneous symmetry breaking \& Nambu-Goldstone theorem}
\section{Reviewing Ginzburg-Landau theory *}
\section{Introducing category theory \& higher groups as a tool *}
\section{Formulating higher gauge theories}
\end{document}
