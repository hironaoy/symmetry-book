\documentclass{jarticle}
\usepackage{appendix}
\usepackage{graphicx} % Required for inserting images
\usepackage{graphics}
\usepackage[margin = 25mm]{geometry}
\usepackage{mathrsfs, mathtools}
\usepackage{amsmath, amsfonts, amsthm, amssymb}
\usepackage{bm}
\usepackage{physics}
\usepackage{tensor}
\usepackage{comment}
\usepackage{etoolbox}
\usepackage{tikz-cd}
\usepackage{fancyhdr}
\usepackage{slashed}

%% \usepackage{lxfonts}

\theoremstyle{definition}
\newtheorem{theorem}{theorem}[section]
\newtheorem{definition}{Definition}[section]
\newtheorem{proposition}{Proposition}[section]

\numberwithin{equation}{section}
\renewcommand{\baselinestretch}{1.1}
\renewcommand{\mapsto}{\longmapsto}

\renewcommand{\appendixname}{Appendix}
\usepackage[colorlinks=false, linkcolor=black, urlcolor=blue, citecolor=blue]{hyperref}
\usepackage{ulem}

\usepackage{titlesec}
\titleformat{\section}
            [hang]
            {\bfseries}
            {\thesection}
            {1em}
            {}
\titleformat{\subsection}
            [hang]
            {\bfseries}
            {\thesubsection}
            {1em}
            {}


\newcommand{\diag}{\mathrm{diag}\,}
\newcommand{\st}{\quad \mathrm{s.t.} \quad}
\newcommand{\ob}{\mathrm{Ob}\,}

\title{{\bf 研究日記:一般化対称性とその周辺}}
\author{大阪大学大学院理学研究科物理学専攻 大和 寛尚}
\date{}

% \newcounter{defcount}
% \setcounter{defcount}{1}
% \renewenvironment{definition}{
%   \par
%   \vspace{2mm}
%   \noindent
%   \uline{Definition \thesection.\thedefcount}
% }{
%   \stepcounter{defcount}
%   \par
%   \vspace{2mm}
%   \aftergroup\noindent\ignorespacesafterend
% }

% \newcounter{propcount}
% \setcounter{propcount}{1}
% \renewenvironment{proposition}{
%   \par
%   \vspace{2mm}
%   \noindent
%   \uline{Proposition \thesection.\thepropcount}
% }{
%   \stepcounter{propcount}
%   \par
%   \vspace{2mm}
%   \aftergroup\noindent\ignorespacesafterend
% }

% \newenvironment{example}{
%   \par
%   \vspace{2mm}
%   \noindent
%   \uline{Example}
% }{
%   \par
%   \vspace{2mm}
%   \aftergroup\noindent\ignorespacesafterend
% }

\begin{document}
\maketitle
\tableofcontents

\section{高次形式対称性の定義}
このセクションの内容は主に以下の文献に依拠する。
\begin{enumerate}
\item 日高 義将,高次対称性入門
\item 日高 義将,対称性の自発的破れ入門
\end{enumerate}
\subsection{従来の対称性からの出発}
具体的な例から出発することにする。いま、$D = d + 1$次元自由フェルミオン系を考える。作用$S$は次で与えられる。
\begin{align}
  S[\psi] = -\int_{\mathscr{M}} \dd^D x \, \bar{\psi}(x)\qty(\gamma^\mu\partial_\mu + m) \psi(x)
\end{align}
ただし、$\psi(x)$はディラック場、$\bar{\psi}(x) = i\psi^\dagger(x)\gamma^0$であり、$\mathscr{M} = \mathbb{R} \times \mathbb{R}^d$はミンコフスキー計量をもつ時空多様体である。

この系において$U(1)$対称性変換のネーター・カレント$j^\mu$と対応するネーター・チャージ$Q$を計算すると
\begin{align}
  j^\mu(x) = -i\bar{\psi}(x)\gamma^\mu\psi(x), \quad Q(x^0) = -i\int_{\mathbb{R}^d} \dd^d x\, \bar{\psi}(x)\gamma^0 \psi(x)
\end{align}
である。$Q$は$U(1)$変換の生成子のヒルベルト空間上の表現である。つまり、次が成り立つ。
\begin{align}
  e^{i\theta Q(x^0)} \psi(x) e^{-i\theta Q(x^0)} = e^{i\theta} \psi(x), \quad
  \qty[iQ(x^0), \psi(x)] = e^{i\theta} \psi(x).
\end{align}


\section{$U(1)$ゲージ理論における高次形式対称性}

\end{document}
