\documentclass{article}
\usepackage{appendix}
\usepackage{graphicx} % Required for inserting images
\usepackage{graphics}
\usepackage[margin = 25mm]{geometry}
\usepackage{mathrsfs, mathtools}
\usepackage{amsmath, amsfonts, amsthm, amssymb}
\usepackage{bm}
\usepackage{physics}
\usepackage{tensor}
\usepackage{comment}
\usepackage{etoolbox}
\usepackage{tikz-cd}
\usepackage{fancyhdr}
\usepackage{slashed}

%% \usepackage{lxfonts}

\theoremstyle{definition}
\newtheorem{theorem}{theorem}[section]
\newtheorem{definition}{Definition}[section]
\newtheorem{proposition}{Proposition}[section]

\numberwithin{equation}{section}
\renewcommand{\baselinestretch}{1.3}
\renewcommand{\mapsto}{\longmapsto}

\renewcommand{\appendixname}{Appendix}
\usepackage[colorlinks=false, linkcolor=black, urlcolor=blue, citecolor=blue]{hyperref}
\usepackage{ulem}

\usepackage{titlesec}
\titleformat{\section}
            [hang]
            {\bfseries}
            {\thesection}
            {1em}
            {}
\titleformat{\subsection}
            [hang]
            {\bfseries}
            {\thesubsection}
            {1em}
            {}


\newcommand{\diag}{\mathrm{diag}\,}
\newcommand{\st}{\quad \mathrm{s.t.} \quad}
\newcommand{\ob}{\mathrm{Ob}\,}

% \pagestyle{fancy}
% \fancyhead[L]{\thepage}
% \fancyhead[R]{Topology \& Physics}
% \fancyfoot{}
% \renewcommand{\headrulewidth}{0pt}

\title{{\bf Topology \& Physics}}
\author{HIRONAO YAMATO}
\date{}

\newcounter{defcount}
\setcounter{defcount}{1}
\renewenvironment{definition}{
  \par
  \vspace{2mm}
  \noindent
  \uline{Definition \thesection.\thedefcount}
}{
  \stepcounter{defcount}
  \par
  \vspace{2mm}
  \aftergroup\noindent\ignorespacesafterend
}

\newcounter{propcount}
\setcounter{propcount}{1}
\renewenvironment{proposition}{
  \par
  \vspace{2mm}
  \noindent
  \uline{Proposition \thesection.\thepropcount}
}{
  \stepcounter{propcount}
  \par
  \vspace{2mm}
  \aftergroup\noindent\ignorespacesafterend
}

\newenvironment{example}{
  \par
  \vspace{2mm}
  \noindent
  \uline{Example}
}{
  \par
  \vspace{2mm}
  \aftergroup\noindent\ignorespacesafterend
}

\begin{document}
\tableofcontents

\section{Mathmatical preparations}
We begin by estabilishing the mathematical framework. In dealing with symmetries, higher-form symmetries in particular, differential forms are a very convenient tool. First we will introduce differential forms and in relation to that, we will introduce de Rham cohomology groups.

\subsection{Differential forms}
We will introduce differential forms in a very intuitive way, sacrificing the mathematical rigor. The most part of this section are adapted from \cite{Hidaka}.

\subsubsection{Introduction: Differential forms in 2 dimensional Euclid space}
Here we are dealing with a 2 dimensional Euclid space $M$. And we consider a integral of a vector field $(a_x(x, y), a_y(x, y))$ on a curve $\gamma(t) = (x(t), y(t))$. Here $t \in [0, 1]$. In order to define the integral we first introduce a \textit{1-form} $\omega$:
\begin{align}
  \omega := a_x(x, y) \dd x + a_y(x, y) \dd y.
\end{align}
Then we define the integral as follows.
\begin{align}
  \int_\gamma \omega :=
  \int_0^1 \dd t \, \qty{a_x(x(t), y(t)) \dv{x(t)}{t} + a_y(x(t), y(t)) \dv{y(t)}{t}}.
\end{align}
In the case that $\omega$ is the total derivative, which means that $a_x(x, y) = \partial_x f(x,y)$ and $a_y(x, y) = \partial_y f(x,y)$ with $^\exists f(x, y)$, we write
\begin{align}
  \int_\gamma \omega = \int_\gamma \dd f = f(\gamma(1)) - f(\gamma(0)). \label{1-dim-stokes-lhs}
\end{align}

Next, we introduce $\partial \gamma := \gamma(1) - \gamma(0)$. And we also define a integral
\begin{align}
  \int_{\partial \gamma} f := f(\gamma(1)) - f(\gamma(0)). \label{1-dim-stokes-rhs}
\end{align}
Combining \eqref{1-dim-stokes-lhs} and \eqref{1-dim-stokes-rhs}, we get
\begin{align}
  \int_\gamma \dd f = \int_{\partial \gamma} f.
\end{align}
This is the 1 dimensional version of Stokes's theorem.

Next, we introduce a \textit{2-form} $\dd \omega$ as
\begin{align}
  \dd \omega := \partial_x a_y(x, y) \dd x \wedge \dd y + \partial_y a_x(x, y) \dd y \wedge \dd x.
\end{align}
Here, we introduced the wedge product $\cdot \wedge \cdot$
\begin{align}
  \dd x \wedge \dd y = - \dd y \wedge \dd x.
\end{align}
In the case of $\omega = \dd f$, we get
\begin{align}
  \dd \omega =
  \partial_x \partial_y f(x,y) \dd x \wedge \dd y + \partial_y \partial_x f(x,y) \dd y \wedge \dd x = 0.
\end{align}

\subsubsection{Generalization: Differential forms in $D$ dimensional Euclid space}
Now we generalize the properties we got from the 2 dimensional case to $D$ dimensional case.
\begin{definition}
  We define the wedge product $\cdot \wedge \cdot$ as 
  \begin{align}
    \dd x^\mu \wedge \dd x^\nu = - \dd x^\nu \wedge \dd x^\mu.
  \end{align}
  Introducing the associativity, we generalize this to the case of more than two terms.
  \begin{align}
    \dd x^\mu \wedge \dd x^\nu \wedge \dd x^\rho = - \dd x^\mu \wedge \dd x^\rho \wedge \dd x^\nu
                                                 = - \dd x^\nu \wedge \dd x^\mu \wedge \dd x^\rho.
  \end{align}
\end{definition}
\begin{definition}
  We define a $p$\textit{-form} as
  \begin{align}
    X := \frac{1}{p!} X_{\mu_1 \cdots \mu_p} \dd x^{\mu_1} \wedge \cdots \wedge \dd x^{\mu_p}.
  \end{align}
  And we write the entire set of $p$-form as $\varOmega^p(M)$.
\end{definition}
\begin{definition}
  We define the \textit{exterior derivative} as
  \begin{align}
    \dd : \varOmega^p(M) \longrightarrow \varOmega^{p+1}(M); \quad
    \frac{1}{p!} X_{\mu_1 \cdots \mu_p} \dd x^{\mu_1} \wedge \cdots \wedge \dd x^{\mu_p}
    \mapsto \frac{1}{p!}\partial_\nu X_{\mu_1\cdots\mu_p} \dd x^\nu \wedge \dd x^{\mu_1} \wedge \cdots \wedge \dd x^{\mu_p}.
  \end{align}
\end{definition}


\subsection{de Rham cohomology groups}

\begin{thebibliography}{9}
\bibitem{Hidaka} Yoshimasa Hidaka,
  \href{https://ribf.riken.jp/~hidaka/yh/slide/hidaka_higher_form.pdf}
       {\textit{Introduction to Higher Symmetries}}.
\end{thebibliography}

\end{document}
