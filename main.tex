\documentclass{article}
\usepackage{appendix}
\usepackage{graphicx} % Required for inserting images
\usepackage{graphics}
\usepackage[margin = 25mm]{geometry}
\usepackage{mathrsfs, mathtools}
\usepackage{amsmath, amsfonts, amsthm, amssymb}
\usepackage{bm}
\usepackage{physics}
\usepackage{tensor}
\usepackage{comment}
\usepackage{etoolbox}
\usepackage{tikz-cd}
\usepackage{fancyhdr}
\usepackage{slashed}

%% \usepackage{lxfonts}

\theoremstyle{definition}
\newtheorem{theoremph}{Theorem$^\mathrm{(ph)}$}[section]
\newtheorem{theorem}{Theorem}[section]
\newtheorem{definitionph}{Definition$^\mathrm{(ph)}$}[section]
\newtheorem{definition}{Definition}[section]
\newtheorem{propositionph}{Proposition$^\mathrm{(ph)}$}[section]
\newtheorem{proposition}{Proposition}[section]
\newtheorem{factph}{Fact$^\mathrm{(ph)}$}[section]
\newtheorem{fact}{Fact}[section]
\newtheorem{remark}{Remark}[section]
\newtheorem{observationph}{Observation$^{(\mathrm{ph})}$}[section]
\newtheorem{observation}{Observation}[section]

\numberwithin{equation}{section}
\renewcommand{\baselinestretch}{1.3}
\renewcommand{\mapsto}{\longmapsto}

\renewcommand{\appendixname}{Appendix}
\usepackage[dvipdfmx, colorlinks=true, linkcolor=black, urlcolor=black, citecolor=black]
           {hyperref}
\usepackage{pxjahyper}
\usepackage{ulem}

\usepackage{titlesec}
\titleformat{\section}
            [hang]
            {\bfseries}
            {\thesection}
            {1em}
            {}
\titleformat{\subsection}
            [hang]
            {\bfseries}
            {\thesubsection}
            {1em}
            {}

            \newcommand{\diag}{\mathrm{diag}\,}
\newcommand{\st}{\quad \mathrm{s.t.} \quad}
\newcommand{\ob}{\mathrm{Ob}\,}

\newcommand{\fivarphi}{\int \mathscr{D} \varphi\,}
\newcommand{\fipsi}{\int \mathscr{D} \varphi\,}
\newcommand{\fidim}{\int \qty(\prod_{a = 1}^{N} \mathscr{D}\varphi_a) \, }
\newcommand{\intst}{\int \dd^Dx \,}
\newcommand{\del}{\partial_\mu}
\newcommand{\lag}{\mathscr{L}}
\newcommand{\link}{\mathrm{Link}}

\title{{\large {\bf Notes for master's thesis}}\\{\bf Foundations \& applications of generalised symmetries}}
\author{大阪大学大学院理学研究科物理学専攻 大和 寛尚}
\date{}

\begin{document}
\maketitle

\tableofcontents
\footnote{The sections with asterisk `` * '' are materials that is not directly related to generalised symmetries but supplemental.}

\section{Foundations}
\subsection{Conventional symmetries *}
Most parts of this section are adopted from the matirials below.
\begin{enumerate}
\item [\cite{Peskin:1995ev}]
M.~E.~Peskin and D.~V.~Schroeder,
``An Introduction to quantum field theory''
\item [\cite{Srednicki:2007qs}]
M.~Srednicki,
``Quantum field theory''
\item [\cite{Gomes:2023ahz}]
P.~R.~S.~Gomes,
``An introduction to higher-form symmetries'', Section 3
\end{enumerate}

\subsubsection{Schwinger-Dyson equation \& conversation laws}
\begin{propositionph}
  We have the following relation, which is well-known as Schwinger-Dyson equation.
  \begin{align}
    \ev{\frac{\partial S[\varphi]}{\partial \varphi(x)} \varphi(x_1) \cdots \varphi(x_n)} = i \sum_{i = 1}^{n} \ev{\varphi(x_1) \cdots \delta^{(D)}(x - x_i) \cdots \varphi(x_n)}.
  \end{align}  
\end{propositionph}
\begin{proof}
  First of all, we prove the general relation
  \begin{align}
    \int \mathscr{D} \varphi \, \frac{\delta}{\delta \varphi(x)} \qty(F[\varphi] e^{iS[\varphi]}) = 0.
  \end{align}
  We start from the obvious relation
  \begin{align}
    \int \mathscr{D} \varphi \, F[\varphi] e^{iS[\varphi]} = \int \mathscr{D} \varphi' \, F[\varphi'] e^{iS[\varphi']},
  \end{align}
  which is just a renaming of a dummy variable of the integration. Now we perform a change of variables $\varphi'(x) \mapsto \varphi(x) + \varepsilon(x)$. Then we have
  \begin{align}
    \int \mathscr{D}\varphi \, F[\varphi] e^{iS[\varphi]}
    &= \fivarphi F[\varphi + \varepsilon]e^{iS[\varphi + \varepsilon]} \notag \\
    &= \fivarphi \qty{F[\varphi] + \intst \frac{\delta F[\varphi]}{\delta \varphi(x)} \varepsilon(x)}
      \exp\qty{iS[\varphi] + i \intst \frac{\delta S[\varphi]}{\delta \varphi(x)} \varepsilon(x)} \notag \\
    &= \fivarphi \qty{F[\varphi] + \intst \frac{\delta F[\varphi]}{\delta \varphi(x)} \varepsilon(x)}
      \qty{1 + i\intst \frac{\delta S[\varphi]}{\delta \varphi(x)}\varepsilon(x)}e^{iS[\varphi]} \notag \\
    &= \fivarphi F[\varphi] e^{iS[\varphi]} + \fivarphi \intst
      \qty{\frac{\delta F[\varphi]}{\delta\varphi(x)} e^{iS[\varphi]} + F[\varphi] i\frac{\delta S[\varphi]}{\delta \varphi(x)} e^{iS[\varphi]}}\varepsilon(x),
      \quad ^\forall \varepsilon(x),
  \end{align}
  which leads us to the identity
  \begin{align}
    \fivarphi \qty{\frac{\delta F[\varphi]}{\delta\varphi(x)} e^{iS[\varphi]} + F[\varphi] i\frac{\delta S[\varphi]}{\delta \varphi(x)} e^{iS[\varphi]}} = 0.
  \end{align}
  Now, by substituting $F[\varphi]$ with $\varphi(x_1) \cdots \varphi(x_n)$ we get
  \begin{align}
    i\fivarphi \frac{\delta S[\varphi]}{\delta\varphi(x)}  \varphi(x_1) \cdots \varphi(x_n) e^{iS[\varphi]}
    = - \sum_{i = 1}^{n} \fivarphi \varphi(x_1) \cdots \delta^{(D)}(x - x_i) \cdots \varphi(x_n) e^{iS[\varphi]},
  \end{align}
  which we simply denote
  \begin{align}
    \ev{\frac{\delta S[\varphi]}{\delta\varphi(x)}  \varphi(x_1) \cdots \varphi(x_n)}
    = i \sum_{i = 1}^{n} \ev{\varphi(x_1) \cdots \delta^{(D)}(x - x_i) \cdots \varphi(x_n)}.
  \end{align}
\end{proof}

\begin{remark}
  Schwinger-Dyson equation is the generalisation of Ward-Takahashi identity.
\end{remark}

\begin{propositionph}
  We have following relation, which is a correlation function representation of the conversation law of Noether current.
  \begin{align}
    \ev{\qty{\del j^\mu(x)} \varphi_{a_1}(x_1) \cdots \varphi_{a_n}(x_n)}
    = -i\sum_{i = 0}^{n} \ev{\varphi_{a_1}(x_1) \cdots \delta^{(D)}(x - x_i) \delta \varphi_{a_i}(x_i) \cdots \varphi_{a_n}(x_n)}. \label{noether-takahashi-id}
  \end{align}
\end{propositionph}
\begin{proof}
  Now, we consider the transformation
  \begin{align}
    \varphi_a(x) \mapsto \varphi_a'(x) = \varphi_a(x) + \delta_{\varepsilon(x)} \varphi_a(x) = \varphi_a(x) + \varepsilon(x) g_a(x).
  \end{align}
  We first prove the relation
  \begin{align}
    \delta_{\varepsilon(x)} S[\varphi] = - \intst \varepsilon(x) \del \qty{\sum_{a = 1}^{N} g_a(x) \frac{\partial \lag}{\partial(\partial_\mu \varphi_a(x))} - K^\mu[\varphi]}
    = - \intst \varepsilon(x) \del j^\mu(x)
  \end{align}
  where we defined $j^\mu(x)$ as
  \begin{align}
    j^\mu(x) = \sum_{a = 1}^{N} g_a(x) \frac{\partial \lag}{\partial(\partial_\mu \varphi_a(x))} - K^\mu[\varphi].
  \end{align}
  The direct calculation of $\delta_{\varepsilon(x)} S[\varphi]$ is as follows.
  \begin{align}
    \delta_{\varepsilon(x)} S[\varphi]
    &= \intst \delta_{\varepsilon(x)} \lag[\varphi, \del \varphi]
     = \intst \qty{\lag[\varphi(x) + \varepsilon(x)\varphi(x), \del \qty{\varphi(x) + \varepsilon(x)g(x)}] - \lag[\varphi(x), \del \varphi(x)]} \notag \\
    &= \intst \qty{\lag[\varphi(x) + \varepsilon(x)\varphi(x), \del \varphi(x) + \del \varepsilon(x) g(x) + \varepsilon(x) \del g(x)] - \lag[\varphi(x), \del \varphi(x)]} \notag \\
    &= \intst \qty{\varepsilon(x)\sum_{a = 1}^{N}g_a(x) \pdv{\lag}{\varphi_a(x)} + \varepsilon(x)\sum_{a = 1}^{N}\del g_a(x)\pdv{\lag}{(\del \varphi_a(x))}
      + \del \varepsilon(x) \sum_{a = 1}^{N} g_a(x) \pdv{\lag}{(\del\varphi_a(x))}} \notag \\
    &= \intst \varepsilon(x) \del K^\mu(x) + \intst \del \varepsilon(x) \sum_{a = 1}^{N} g_a(x) \pdv{\lag}{(\del\varphi_a(x))}
  \end{align}
  Here, we assumed
  \begin{align}
    \sum_{a = 1}^{N}g_a(x) \pdv{\lag}{\varphi_a(x)} + \sum_{a = 1}^{N}\del g_a(x)\pdv{\lag}{(\del \varphi_a(x))} = \del K^\mu[\varphi(x)], \quad ^\exists K^\mu[\varphi(x)].
  \end{align}
  Integrating the second term by parts, we get 
  \begin{align}
    \delta_{\varepsilon(x)} S[\varphi] = \intst \varepsilon(x) \del K^\mu(x) - \intst \varepsilon(x) \del \qty{ g_a(x)\pdv{\lag}{(\del \varphi_a(x))}}
    = - \intst \varepsilon(x) \del j^\mu(x).
  \end{align}
  Now, we move on to the proof of the relation in consider. We start from the trivial relation
  \begin{align}
    \fidim \varphi_{a_1}(x_1) \cdots \varphi_{a_n}(x_n) e^{iS[\varphi]}
    = \int \qty(\prod_{a = 1}^{N} \mathscr{D}\varphi'_a)\, \varphi'_{a_1}(x_1) \cdots \varphi'_{a_n}(x_n) e^{iS[\varphi]}.
  \end{align}
  Here, we perform the change of variable $\varphi'_a(x) = \varphi_a(x) + \varepsilon(x) g_a(x)$ and we get
  \begin{align}
    \fidim \varphi_{a_1}(x_1) \cdots \varphi_{a_n}(x_n) e^{iS[\varphi]}
    &= \fidim \varphi_{a_1}(x_1) \cdots \varphi_{a_n}(x_n) e^{iS[\varphi]} \notag \\
    &\quad + i\fidim \varphi_{a_1}(x_1) \cdots \varphi_{a_n}(x_n) \qty{- \intst \varepsilon(x) \del j^\mu(x)}e^{iS[\varphi]} \notag \\
    &\quad + \sum_{i = 1}^{n} \fidim \varphi_{a_1}(x_1) \cdots \varepsilon(x_i) g_{a_i}(x_i) \cdots \varphi_{a_n}(x_n)e^{iS[\varphi]},
  \end{align}
  which leads us to
  \begin{align}
    &-i\fidim \varphi_{a_1}(x_1) \cdots \varphi_{a_n}(x_n) \qty{\intst \varepsilon(x) \del j^\mu(x)}e^{iS[\varphi]} \notag \\
    & \quad\quad\quad\quad = \sum_{i = 1}^{n} \fidim \varphi_{a_1}(x_1) \cdots \varepsilon(x_i) g_{a_i}(x_i) \cdots \varphi_{a_n}(x_n)e^{iS[\varphi]} = 0.
  \end{align}
  Now, we substitute $\varepsilon(x_i)$ in RHS with a trivial relation ${\displaystyle \varepsilon(x_i) = \intst \varepsilon(x) \delta^{(D)}(x - x_i)}$, we get
  \begin{align}
    &-i\intst \varepsilon(x) \fidim \varphi_{a_1}(x_1) \cdots \varphi_{a_n}(x_n) \qty{\del j^\mu(x)}e^{iS[\varphi]} \notag\\
    & \quad\quad\quad\quad = \intst \varepsilon(x) \sum_{i = 1}^{n} \fidim \varphi_{a_1}(x_1) \cdots \delta^{(D)}(x - x_i) g_{a_i}(x_i) \cdots \varphi_{a_n}(x_n)e^{iS[\varphi]},
      \quad ^\forall \varepsilon(x).
  \end{align}
  Then we get
  \begin{align}
    i\fidim \varphi_{a_1}(x_1) \cdots \varphi_{a_n}(x_n) \qty{\del j^\mu(x)}e^{iS[\varphi]}
    = \sum_{i = 1}^{n} \fidim \varphi_{a_1}(x_1) \cdots \delta^{(D)}(x - x_i) g_{a_i}(x_i) \cdots \varphi_{a_n}(x_n)e^{iS[\varphi]},
  \end{align}
  which means that
  \begin{align}
    \ev{\varphi_{a_1}(x_1) \cdots \varphi_{a_n}(x_n) \qty{\del j^\mu(x)}} = -i\sum_{i = 0}^{n}\ev{\varphi_{a_1}(x_1) \cdots \delta^{(D)}(x - x_i) g_{a_i}(x_i) \cdots \varphi_{a_n}(x_n)}.
    \label{corr-conv-law}
  \end{align}
\end{proof}

\subsubsection{Reformulation in the language of differential forms}
\begin{factph}
  Mathematically speaking, Noether current $j^\mu(x)$ are coefficients of a vector field on the spacetime manifold $\mathscr{M}$, which is a section of the tangent bundle $T\mathscr{M}$. And equivaletly $j_\mu(x)$ are coefficients of a covector field of a section of the cotangent bundle $T^*\mathscr{M}$. Let us stick to the representation on cotangent bundle in this notes, then we can write
  \begin{align}
    j = j_\mu(x) \dd x^\mu.
  \end{align}
  And obviously, we have the Hodge dual of it.
  \begin{align}
    *j = \frac{1}{(D - 1)!}j_{\mu_0}(x) \tensor{\epsilon}{^{\mu_0}_{\mu_1\cdots\mu_{D-1}}} \dd x^{\mu_1} \wedge \cdots \wedge \dd x^{\mu_{D-1}}.
  \end{align}
\end{factph}

\begin{propositionph}
  The conversation law $\del j^\mu(x) = 0$ can be written as
  \begin{align}
    \dd *j = 0.
  \end{align}
\end{propositionph}

\begin{proof}
  We will see the equivalence of the two representations $\del j^\mu = \partial^\mu j_\mu = 0$ and $\dd * j = 0$. We start from
  \begin{align}
    \dd * j
    &= \frac{1}{(D - 1)!}\partial_\alpha j_{\mu_0}(x)
      \tensor{\epsilon}{^{\mu_0}_{\mu_1\cdots\mu_{D-1}}} \dd x^{\alpha} \wedge \dd x^{\mu_1} \wedge \cdots \wedge \dd x^{\mu_{D-1}} = \frac{1}{(D - 1)!}\partial_\alpha j_{\mu_0}(x)
      \tensor{\epsilon}{^{\mu_0}_{\mu_1\cdots\mu_{D-1}}}\tensor{\epsilon}{^{\alpha\mu_1\cdots\mu_{D-1}}} \notag\\
    &= (-1)^{D-1}\partial_\alpha j_{\mu_0}(x)\eta^{{\mu_0}\alpha} \dd x^{0} \wedge \cdots \wedge \dd x^{D-1} = (-1)^{D-1}\partial^\mu j_\mu(x) \dd x^0 \wedge \cdots \dd x^{D-1}.
  \end{align}
  Now, if we assume $\dd * j = 0$, obviously $\partial_\mu j^\mu(x) = 0$ because $\dd x^0 \wedge \cdots \wedge \dd x^{D-1} \ne 0$. Also if we assume $\partial_\mu j^\mu(x) = 0$, obviously $\dd * j = 0$ from the relation we just derived.
\end{proof}

\begin{remark}
  In the Euclidean theory, which we will be using so often in this note, we simply have
  \begin{align}
    \dd * j = \del j^\mu(x) \dd x^0 \wedge \cdots \wedge \dd x^{D-1}.
  \end{align}
\end{remark}

\begin{propositionph}
  The definition of Noether charge ${\displaystyle Q = \int \dd^{D-1}x \, j_0(x)}$ is translated to differential forms as follows.
  \begin{align}
    Q = \int_\Sigma * j, \label{noether-charge}
  \end{align}
  where, $\Sigma$ is the spatial submanifold of the spacetime manifold $\mathscr{M}$.
\end{propositionph}

\begin{proof}
  We start by rewriting the integral.
  \begin{align}
    Q &= \int_{\Sigma} *j = \int_{\Sigma} \frac{1}{(D - 1)!}j_{\mu_0}(x) \tensor{\epsilon}{^{\mu_0}_{\mu_1\cdots\mu_{D-1}}} \dd x^{\mu_1} \wedge \cdots \wedge \dd x^{\mu_{D-1}}
    \notag \\
      &= \int_{\Sigma} \frac{1}{(D - 1)!}j_{\mu_0}(x) \tensor{\epsilon}{^{\mu_0}_{\mu_1\cdots\mu_{D-1}}}\tensor{\epsilon}{^{0\mu_1\cdots\mu_{D-1}}}
        \dd x^{1} \wedge \cdots \wedge \dd x^{D-1} = \int_{\Sigma} j_{\mu_0}(x) (-1)^{D-1}\eta^{\mu_0 0} \dd x^{1} \wedge \cdots \wedge \dd x^{D-1}. \label{noether-charge}
  \end{align}
  Obviously, \eqref{noether-charge} $= 0$ for $\mu_0 \ne 0$. Then we have
  \begin{align}
    Q = Q(\Sigma) = (-1)^{D-1}  \int_{\Sigma} j_{0}(x) \, \dd x^{1} \wedge \cdots \wedge \dd x^{D-1},
  \end{align}
  which has the same meaning as ${\displaystyle \int \dd^{D-1}x}\, j_0(x)$.\footnote{Here, we ignore the factor $(-1)^{D-1}$ because it does not affect the physical invariant.}
\end{proof}

\begin{propositionph}
  The conversation law in a correlation function representation is interpreted as follows.
  \begin{align}
    \ev{(\dd * j) \varphi_{a_1}(x_1) \cdots \varphi_{a_n}(x_n)}
    = -i \sum_{i = 0}^{n} \ev{\varphi_{a_1}(x_1) \cdots \delta^{(D)}(x - x_i)g_{a_i}(x_i) \cdots \varphi_{a_n}(x_n)} \dd x^0 \wedge \cdots \wedge \dd x^{D-1}.
    \label{noether-takahashi-id-df}
  \end{align}
\end{propositionph}

\begin{proof}
  By simply putting $\dd x^0 \wedge \cdots \wedge \dd x^{D-1}$ on both sides of \eqref{noether-takahashi-id}, we get \eqref{noether-takahashi-id-df}.
\end{proof}

\begin{definitionph}
  We say that, in the case of conventional symmetries, the operator $\mathscr{O}(\mathscr{M})$ defined on some manifold $\mathscr{M}$ is topological if $\ev{\mathscr{O}(\mathscr{M}) \varphi(x_1) \cdots \varphi(x_n)}$ is invariant under the deformation of $\mathscr{M}$.
\end{definitionph}

\begin{propositionph}
  We have the following relation.
  \begin{align}
    \ev{Q(\Sigma)\varphi_{a_1}(x_1) \cdots \varphi_{a_n}(x_n)}
    = -i \sum_{i = 0}^{n}  \link(\Sigma, x_i)\ev{\varphi_{a_1}(x_1) \cdots g_{a_i}(x_i) \cdots \varphi_{a_n}(x_n)}, \label{noether-charge-cf}
  \end{align}
  where we defined the link number as
  \begin{align}
    \link(\Sigma, x_i) = \int_{\varOmega_{\Sigma}} \dd x^0 \wedge \cdots \dd x^{D-1} \delta^{(D)} (x - x_i), \quad \partial\varOmega_{\Sigma} = \Sigma. \label{link-number}
  \end{align}
\end{propositionph}

\begin{proof}
  We just integrate the both sides of \eqref{noether-takahashi-id-df} in terms of $x$ on $\varOmega_\Sigma$.
  \begin{align}
    \text{LHS of } \eqref{noether-takahashi-id-df} &= \int_{\varOmega_\Sigma}\ev{(\dd * j) \varphi_{a_1}(x_1) \cdots \varphi_{a_n}(x_n)} \notag\\
                                                   &= \ev{\int_{\varOmega_\Sigma}\qty(\dd * j) \varphi_{a_1}(x_1) \cdots \varphi_{a_n}(x_n)}\overset{\text{Stokes's theorem}}{=}
                                                     \ev{\int_{\Sigma} * j \, \varphi_{a_1}(x_1) \cdots \varphi_{a_n}(x_n)} \notag\\
                                                   &\overset{\eqref{noether-charge}}{=} \ev{\int_{\Sigma} Q(\Sigma) \, \varphi_{a_1}(x_1) \cdots \varphi_{a_n}(x_n)} \\
    \text{RHS of } \eqref{noether-takahashi-id-df} &= -i \sum_{i = 0}^{n} \int_{\varOmega_\Sigma} \delta^{(D)}(x - x_i)
                                                     \ev{\varphi_{a_1}(x_1) \cdots g_{a_i}(x_i) \cdots \varphi_{a_n}(x_n)} \dd x^0 \wedge \cdots \wedge \dd x^{D-1} \notag \\
                                                   &= -i \sum_{i = 0}^{n} \int_{\varOmega_\Sigma} \delta^{(D)}(x - x_i)
                                                     \ev{\varphi_{a_1}(x_1) \cdots g_{a_i}(x_i) \cdots \varphi_{a_n}(x_n)} \dd x^0 \wedge \cdots \wedge \dd x^{D-1} \notag \\
                                                   &= -i \sum_{i=0}^{n} \link(\Sigma, x_i)\ev{\varphi_{a_1}(x_1) \cdots g_{a_i}(x_i) \cdots \varphi_{a_n}(x_n)}
  \end{align}
  This proves the proposition.
\end{proof}
\begin{remark}
  The link number \eqref{link-number} is topological under the deformation that does not cross the point $x_i$.
\end{remark}

For simplicity, we will be considering $n = 1$ case only for the present. \eqref{noether-takahashi-id-df} and \eqref{noether-charge-cf} are
\begin{align}
  \ev{\qty{\dd * j(x)} \varphi(y)} = -i \delta^{(D)}(x - y) \ev{g(y)} \dd x^0 \wedge \cdots \wedge \dd x^{D-1}, \quad\ev{Q(\Sigma) \varphi(y)} = -i \link(\Sigma, y) \ev{g(y)}.
\end{align}

\begin{theoremph}
  Noether charge is topological under the deformation $\varOmega_\Sigma \mapsto \varOmega'_\Sigma = \varOmega_\Sigma \cup \varOmega_0$ such that $y$ does not belong to $\varOmega_0$. This is the topological expression of the conversation law of Noether charge.
\end{theoremph}

\begin{proof}
  Under the defirmation $\varOmega_\Sigma \mapsto \varOmega'_\Sigma = \varOmega_\Sigma \cup \varOmega_0$, Nother charge transforms as follows.
  \begin{align}
    \ev{Q(\Sigma + \partial \varOmega_0)\varphi(y)}= \int_{\varOmega_\Sigma \cup \varOmega_0} \ev{\qty{\dd * j(x)}\varphi(y)}
    = \underbrace{\int_{\varOmega_\Sigma} \ev{\qty{\dd * j(x)} \varphi(y)}}_{\ev{Q(\Sigma)\varphi(y)}} + \int_{\varOmega_0} \ev{\qty{\dd * j(x)} \varphi(y)}.
  \end{align}
  Here, since $\varOmega_0$ does not include the point $y$ and $\dd * j(x) = 0$, the second term vanishes. Then we get
  \begin{align}
    \ev{Q(\Sigma + \partial \varOmega_0) \varphi(y)} = \ev{Q(\Sigma)\varphi(y)}.
  \end{align}
  This proves the theorem.
\end{proof}

\begin{propositionph}
  We have the following relation.
  \begin{align}
    \ev{U(\Sigma)\varphi(y)} = R \ev{\varphi(y)}
  \end{align}
  Here, we defined
  \begin{align}
    U(\Sigma) = e^{i\epsilon Q}, \quad R = e^{i\epsilon T},
  \end{align}
  where $\epsilon$ is the arbitary real number and $T$ is the representation on the vector space spanned by the field $\varphi(y)$ of the generator of the symmetry we are considering.
\end{propositionph}

\begin{proof}
  %% TODO: Write the proof.
\end{proof}

\begin{remark}
  $Q(\Sigma)$ is the representation on the Hilbert space of the generator of the symmetry.
\end{remark}

\subsubsection{New perspective on symmetries}
\begin{observationph}
  It may be possible to generalise symmetries as
  \begin{align}
    \text{symmetry generator} = \text{topological operator}.
  \end{align}
\end{observationph}
\subsection{Definition of higher-form symmetries}
Most parts of this section are adopted from the materials below.
\begin{enumerate}
\item [\cite{Gomes:2023ahz}] P.~R.~S.~Gomes,
``An introduction to higher-form symmetries''
\end{enumerate}

\subsubsection{Higher gauge theory}

\section{Applications}
\subsection{Ginzburg-Landau theory *}

\nocite{*}
% \bibliographystyle{plain}
% \bibliography{refs}

\begin{thebibliography}{99}
%\cite{Peskin:1995ev}
\bibitem{Peskin:1995ev}
M.~E.~Peskin and D.~V.~Schroeder,
%``An Introduction to quantum field theory,''
Addison-Wesley, 1995,
ISBN 978-0-201-50397-5, 978-0-429-50355-9, 978-0-429-49417-8
doi:10.1201/9780429503559
%1928 citations counted in INSPIRE as of 17 Sep 2025

%\cite{Srednicki:2007qs}
\bibitem{Srednicki:2007qs}
M.~Srednicki,
%``Quantum field theory,''
Cambridge University Press, 2007,
ISBN 978-0-521-86449-7, 978-0-511-26720-8
doi:10.1017/CBO9780511813917
% 311 citations counted in INSPIRE as of 17 Sep 2025

%\cite{Gomes:2023ahz}
\bibitem{Gomes:2023ahz}
P.~R.~S.~Gomes,
%``An introduction to higher-form symmetries,''
SciPost Phys. Lect. Notes \textbf{74} (2023), 1
doi:10.21468/SciPostPhysLectNotes.74
[arXiv:2303.01817 [hep-th]].
% 115 citations counted in INSPIRE as of 17 Sep 2025


\end{thebibliography}
\end{document}
